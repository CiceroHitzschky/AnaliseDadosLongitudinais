\section{Dataset Riesby}
\subsection{Sobre o Dataset}
\begin{frame}{Dataset Riesby}
    \begin{block}{Sobre o dataset}
    	\begin{itemize}
    		\justifying
    		\item O dataset Riesby representa um ensaio clínico psiquiátrico longitudinal descrito em Reisby et al. (1977) para tratamento de depressão.
    		\item O estudo focou na relação longitudinal entre os níveis plasmáticos de imipramina (IMI) e desipramina (DMI) e a resposta clínica em 66 pacientes internados com depressão é a mudança nas pontuações de depressão semana a semana.
    		\item Como a imipramina se biotransforma no metabólito ativo desmetilimipramina (ou desipramina), a medição da desipramina também foi feita neste estudo.
    	\end{itemize}
    \end{block}
\end{frame}

\begin{frame}{Desenho do Estudo}
	\begin{block}{Fase Inicial}
		Período de Placebo 
	\end{block}
	\begin{block}{Tratamento}
		Doses de 225mg/dia de imipramina por 4 semanas.
	\end{block}
	\begin{block}{Avaliação}
		Escala de classificação de depressão de Hamilton (Hamilton, 1960).
	\end{block}
	\begin{block}{Medições}
		Nível plasmático de imipramina (IMI) e seu metabólito desipramina (DMI) medidos no final de cada semana de tratamento.
	\end{block}
\end{frame}

\begin{frame}{Desenho do Estudo}
	\begin{block}{Coleta de dados}
		\begin{itemize}
			\item Sexo
			\item Diagnóstico de Depressão: Endógena ou Reativa (Não endógena).
		\end{itemize}
	\end{block}
\end{frame}

\begin{frame}{Desenho do Estudo}
	\begin{block}{Número de Participantes}
		Um total de 66 indivíduos sendo a variação por semana dada por:
		\begin{itemize}
			\item Semana 0: 61 participantes.
			\item Semana 1: 63 participantes.
			\item Semana 2: 65 participantes.
			\item Semana 3: 65 participantes.
			\item Semana 4: 63 participantes.
			\item Semana 5: 58 participantes.
		\end{itemize}
	\end{block}
\end{frame}

\begin{frame}{Data set}
	\begin{table}[ht]
		\centering
		\caption{Níveis plasmáticos de imipramina (IMI) e desipramina (DMI) e HDRS score em pacientes com depressão durante o tratamento psiquiátrico.}
		\resizebox{\textwidth}{!}{ % 
			\begin{tabular}{crccccc}
				\toprule
				ID & Score (HDRS) & Semana & Sexo & Endógena & IMI(mg/L) & DMI(mg/L) \\
				\midrule
				101 & -8 & 0 & 0 & 0 & 4,043050 & 4,204690 \\
				101 & -19 & 1 & 0 & 0 & 3,931830 & 4,812180 \\
				101 & -22 & 2 & 0 & 0 & 4,330730 & 4,962840 \\
				101 & -23 & 3 & 0 & 0 & 4,369450 & 4,962840 \\
				103 & -18 & 0 & 1 & 0 & 2,772590 & 5,236440 \\
				 $\vdots$ & $\vdots$ & $\vdots$ & $\vdots$ &  $\vdots$ & $\vdots$ & $\vdots$ \\
				360 & 12 & 3 & 0 & 1 & 3,637590 & 4,844190 \\
				361 & -19 & 0 & 1 & 1 & 4,204690 & 3,784190 \\
				361 & -22 & 1 & 1 & 1 & 4,584970 & 4,234110 \\
				361 & -23 & 2 & 1 & 1 & 4,382030 & 4,189650 \\
				361 & -11 & 3 & 1 & 1 & 4,624970 & 4,189650 \\
				\bottomrule
			\end{tabular}
		}
	\end{table}

\end{frame}

\begin{frame}{Data set}
	\begin{table}[ht]
		\centering
		\caption{Escore HDRS dos pacientes em cada semana de tratamento.}
		\resizebox{0.5\textwidth}{!}{ % 
			\begin{tabular}{crrrr}
				\toprule
				Semana & 0 & 1 & 2 & 3 \\
				ID &  &  &  &  \\
				\midrule
				101 & -8 & -19 & -22 & -23 \\
				103 & -18 & -9 & -18 & -20 \\
				104 & -11 & -16 & -10 & -29 \\
				105 & -6 & -6 & -9 & -13 \\
				$\vdots$ & $\vdots$ & $\vdots$ & $\vdots$ &$\vdots$ \\
				606 & -7 & -7 & -16 & -18 \\
				607 & 0 & -3 & -10 & -26 \\
				608 & -10 & -12 & -21 & -20 \\
				609 & -3 & -11 & -10 & -23 \\
				610 & -1 & -11 & NaN & -23 \\
				\bottomrule
			\end{tabular}
			
		}
	\end{table}
	
\end{frame}

\begin{frame}{Data set}
	\begin{table}[ht]
		\centering
		\caption{Escore HDRS dos pacientes em cada semana de tratamento.}
		\resizebox{0.5\textwidth}{!}{ % 
			\begin{tabular}{crrrr}
				\toprule
				Semana & 0 & 1 & 2 & 3 \\
				ID &  &  &  &  \\
				\midrule
				101 & -8 & -19 & -22 & -23 \\
				103 & -18 & -9 & -18 & -20 \\
				104 & -11 & -16 & -10 & -29 \\
				105 & -6 & -6 & -9 & -13 \\
				$\vdots$ & $\vdots$ & $\vdots$ & $\vdots$ &$\vdots$ \\
				606 & -7 & -7 & -16 & -18 \\
				607 & 0 & -3 & -10 & -26 \\
				608 & -10 & -12 & -21 & -20 \\
				609 & -3 & -11 & -10 & -23 \\
				610 & -1 & -11 & NaN & -23 \\
				\bottomrule
			\end{tabular}
			
		}
	\end{table}
	
\end{frame}
























